\documentclass{article}
\input{imports.tex}
\input{config.tex}

\begin{document}
\noindent \textbf{MA2601-1 Ecuaciones Diferenciales Ordinarias}\\
\textbf{Profesora:} Salome Martínez\\
\textbf{Auxiliar:} Benjamín Vera Vera


\begin{center}
    \Huge{\textbf{Auxiliar 9}}\\
\textit{\large{Transformada de Laplace}}\\
    \normalsize
    \today
\end{center}

\begin{enumerate}
	\item Resuelva utilizando la transformada de Laplace el siguiente problema de valor inicial
\begin{align*}
y'' + 9y &= 2\sen(3t), \quad t > 0 \\
y(0) &= 1 \\
y'(0) &= 0.
\end{align*}
\item \textbf{(una EDO de coeficientes variables)} Sabiendo que la ecuación
$$
	ty'' + y' + ty = 0
$$
tiene una solución $y: [0, \infty) \to \mathbb{R}$ con $y', y''$ de orden exponencial y que cumple $y(0) = 1, y'(0) = 0$:
\begin{enumerate}
	\item Demuestre que si $Y(s) = \mathcal{L}[y](s)$, entonces
$$
	(1+s^2)Y'(s) + sY(s) = 0.
$$
\item Demuestre que $Y(s) = \frac{1}{\sqrt{1+s^2}}$.
\end{enumerate}
\item \textbf{(Laplace para sistemas)} Encuentre utilizando la transformada de Laplace las funciones $u, v$ tales que
\begin{align*}
\frac{\mathrm{d} u}{\mathrm{d} t} + u + \frac{\mathrm{d} v}{\mathrm{d} t} + v &= 1 \\
2 \frac{\mathrm{d} u}{\mathrm{d} t} + u + \frac{\mathrm{d} v}{\mathrm{d} t} &= 0,
\end{align*}
además de la condición inicial $u(0) = 1, v(0) = 0$.
\item La siguiente ecuación es la ecuación de movimiento de una masa unida a un resorte que se libera con velocidad cero cuando está deformado a una distancia $x=1$ desde la posición de equilibrio. Después de $\frac{\pi}{2}$, la masa es golpeada con un martillo que ejerce un impulso instantáneo sobre ella:
$$
	x'' + 9x = -3 \delta_{ \frac{\pi}{2} }, \quad x(0) = 1, \quad x'(0)=0.
$$
Determine la solución $x(t)$ para $t \geq 0$. Describa lo que le ocurre a la masa después de ser golpeada.

\end{enumerate}

\end{document}
