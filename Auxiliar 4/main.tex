\documentclass{article}
\usepackage[letterpaper, rmargin=4em, lmargin=4em, textheight=63em]{geometry}
\usepackage{fancyhdr}
\usepackage[spanish]{babel}
\usepackage[dvipsnames]{xcolor}
\usepackage{graphicx}
\usepackage{wrapfig}
\usepackage{setspace}
\usepackage{hyperref}

\usepackage{amssymb}
\usepackage{amsmath}

\usepackage{charter}
\usepackage{physics}

\usepackage{multicol}
% GEOMETRY
\setlength{\parskip}{1em}
\pagestyle{fancy}
\lhead{Facultad de Ciencias Físicas y Matemáticas}
\rhead{Universidad de Chile}
\cfoot{ }

\renewcommand{\labelenumi}{\normalsize\bfseries P\arabic{enumi}.}
%\renewcommand{\labelenumii}{\normalsize\bfseries (\alph{enumii})}
\renewcommand{\labelenumiii}{\normalsize\bfseries \roman{enumiii})}

% Alfabeto
\newcommand{\A}{\mathcal{A}}
\newcommand{\B}{\mathcal{B}}
\newcommand{\C}{\mathcal{C}}
\newcommand{\E}{\mathcal{E}}
\newcommand{\F}{\mathcal{F}}
\newcommand{\I}{\mathcal{I}}
\newcommand{\K}{\mathcal{K}}
\renewcommand{\L}{\mathcal{L}}
\newcommand{\M}{\mathcal{M}}
\newcommand{\N}{\mathbb{N}}
\renewcommand{\P}{\mathcal{P}}
\newcommand{\Q}{\mathbb{Q}}
\newcommand{\R}{\mathbb{R}}
\renewcommand{\S}{\mathcal{S}}
\newcommand{\T}{\mathcal{T}}
\newcommand{\X}{\mathcal{X}}
\newcommand{\Z}{\mathbb{Z}}

\DeclareMathOperator{\sen}{sen}
\DeclareMathOperator{\senh}{senh}
\DeclareMathOperator{\tg}{tg}
\DeclareMathOperator{\dom}{dom}
\DeclareMathOperator{\dist}{dist}
\DeclareMathOperator{\argmin}{argmin}
\DeclareMathOperator{\Int}{int}

\renewcommand{\epsilon}{\varepsilon}
\renewcommand{\phi}{\varphi}
\newcommand{\dprod}[2]{\langle #1 , #2 \rangle}

\hypersetup{
    colorlinks=true,
    linkcolor=blue,
    filecolor=magenta,      
    urlcolor=blue,
}


\begin{document}
\noindent \textbf{MA2601-1 Ecuaciones Diferenciales Ordinarias}\\
\textbf{Profesora:} Salome Martínez\\
\textbf{Auxiliar:} Benjamín Vera Vera


\begin{center}
    \Huge{\textbf{Auxiliar 4}}\\
\textit{\large{Ecuaciones lineales de orden superior}}\\
    \normalsize
    \today
\end{center}

\begin{enumerate}
	\item \textbf{(Coeficientes constantes)} Entregue la solución general de las siguientes ecuaciones diferenciales:
		\begin{enumerate}
			\item \(y'' - 4y = 0\)
			\item \(y'' + 4y = 0\)
			\item \(y''' + y'' + y' + y = 0\)
			\item \(y''' - y = 0\)
		\end{enumerate}
	\item \textbf{(Fórmula de Abel)} Supongamos una ecuación lineal de segundo orden del tipo
		\begin{equation} \label{eq:var}
			y'' + p(x)y' + q(x) y = 0
		\end{equation}
		y sean \(y_1, y_2\) soluciones de \eqref{eq:var}. Consideremos la cantidad \(W(x)\) definida por
		\[
			W(x) = y_1(x)y_2'(x) - y_1'(x)y_2(x) = \begin{vmatrix}
				y_1(x) & y_2(x) \\ y_1'(x) & y_2'(x)
			\end{vmatrix}.
		\]
		(Cantidad que pronto en cátedra llamaremos \textit{Wronskiano asociado a las soluciones \(y_1, y_2\)}). Encuentre una ecuación diferencial para \(W(x)\) y concluya que el hecho de que \(y_1, y_2\) sean linealmente independientes es una propiedad que no depende del punto inicial.
	\item \textbf{(Reducción de orden)} Consideremos una ecuación diferencial en la forma de \eqref{eq:var}. Suponga que se conoce una solución \(y_1(x)\) no nula a esta ecuación.
		\begin{enumerate}
			\item Definiendo \(y_2(x) = u(x) y_1(x)\) con \(u(x)\) una función a conocer, encuentre una ecuación para \(u\) y con ello la solución general de \eqref{eq:var}.
			\item Aplique esto a la ecuación
				\begin{equation} \label{eq:ec}
					x^2y''(x) - 3xy'(x) + 4y(x) = 0.
				\end{equation}
				Sabiendo que \(y_1(x) = x^2\) es una solución, encuentre otra en el intervalo \((0, \infty)\).
		\end{enumerate}
	\item \textbf{(Ecuación de Euler-Cauchy)} Volviendo a la ecuación \eqref{eq:ec}, cabe preguntarnos cómo se podría haber resuelto la ecuación sin conocer \(y_1(x) = x^2\) en primer lugar. Para ello, considere el cambio de variables \(x = e^z\) y con él, transforme la ecuación a una en términos de \(y(z)\) a coeficientes constantes. Compare la solución con la que se obtuvo anteriormente.

\end{enumerate}

\end{document}
