\documentclass{article}
\input{imports.tex}
\input{config.tex}

\begin{document}
\noindent \textbf{MA2601-1 Ecuaciones Diferenciales Ordinarias}\\
\textbf{Profesora:} Salome Martínez\\
\textbf{Auxiliar:} Benjamín Vera Vera


\begin{center}
    \Huge{\textbf{Auxiliar 3}}\\
\textit{\large{Preparación Control 1}}\\
    \normalsize
    \today
\end{center}

\begin{enumerate}
	\item Encuentre y describa la familia de soluciones de la siguiente ecuación diferencial
		\[
			2xy \dv{y}{x} = a^2 + y^2 - x^2
		\]
\item Consideremos una ecuación diferencial en la forma
\begin{equation} \label{eq:general-bernoulli}
	y'(x) = P(x) F(y) + Q(x)G(y).
\end{equation}
Supongamos, además, que $F, G$ son diferenciables y tales que $\frac{F \cdot G' - G \cdot F'}{G} = c$ es constante.
\begin{enumerate}
	\item Muestre que con un cambio de variables adecuado, es posible transformar \eqref{eq:general-bernoulli} en una ecuación lineal.
	\item Describa un método para resolver la ecuación de Bernoulli:
		\[
			y'(x) = P(x)y + Q(x)y^n.
		\]
	\item Resuelva la ecuación
		\[
			\dv{y}{x} + xy = \sqrt{y}.
		\]
\end{enumerate}
\item Sea \(y_1\) solución de la ecuación de Ricatti
	\begin{equation} \label{eq:ricatti}
		\dv{y}{x} = P(x) + Q(x)y + R(x)y^2.
	\end{equation}
	Pruebe que si \(u\) es solución de la ecuación (de Bernoulli)
	\[
		\dv{u}{x} - (Q + 2Ry_1)u = Ru^2,
	\]
	entonces \(y = y_1 + u\) resuelve \eqref{eq:ricatti}.
\item Considere la curva \(y(x)\) tal que en cada punto \(P=(x_0, y(x_0))\), el segmento tangente a la curva que une \(P\) con el eje \(X\) tiene longitud constante \(a\).
	\begin{enumerate}
		\item Pruebe que la ecuación diferencial de \(y(x)\) viene dada por
			\begin{equation} \label{eq:tractrix}
				y'^2 a^2 = y^2(1 + y'^2).
			\end{equation}
		\item Resuelva la ecuación \eqref{eq:tractrix}.
	\end{enumerate}
\item Un faro, ubicado en una isla en el origen de coordenadas, apunta constantemente hacia un barco pirata con un haz de luz que forma un ángulo $\theta$ con el eje $X$. El barco se escapa de la ubicación del faro con una trayectoria cuya tangente está siempre desfasada $45^\circ$ respecto del ángulo $\theta$. Determinar las posibles trayectorias seguidas por el barco en su escape.

\end{enumerate}

\end{document}
